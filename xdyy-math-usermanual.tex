\documentclass{xdyy-usermanual}

\xdyymanualsetup{
  info = {
    author            = {夏康玮},
    title             = { \pkg{xdyy-math} 用户手册},
    email             = {kangweixia_xdyy@163.com},
    date              = {2022-02-18},
    version           = {0.0.2},
    github-repository = {https://github.com/xkwxdyy/xdyy-math},
    gitee-repository  = {https://gitee.com/xkwxdyy/xdyy-math},
  }
}

\usepackage{xdyy-math}
% \RenewDocumentCommand { \leftbar } { } { }
\usepackage{xchoices}


\begin{document}
\maketitle
\tableofcontents


\section{宏包计划}

本节主要列举想要编写的新功能。

\begin{enumerate}
  \item 细节补充
    \begin{itemize}
      \item 读书笔记中最重要的就是对书中一些地方进行细节的补充和阐述自己的理解
      \item 特别是一些教材喜欢跳步,就需要自己去尝试补充细节,让证明更加完善
    \end{itemize}
    需要涉及到的参数:
    \begin{xchoices}[label-style = arabic]
      \item 补充哪本书的内容,版本等细节
      \item 补充的页码
      \item 补充的原文是什么
      \item 补充了什么
    \end{xchoices}
  \item 勘误
    教材难免会有一些勘误的地方,可以汇总起来方便查找和分享。需要涉及到的参数:
      \begin{enumerate}
        \item 勘误的是什么书还是文献
        \item 勘误的版本?或者出版时间之类的补充细节
        \item 勘误的原文
        \item 勘误后的修改文
        \item 为什么是需要勘误的以及自己的理解
      \end{enumerate}
\end{enumerate}


\section{宏包简介}

自己在记数学笔记的过程中难免会遇到写一些新命令环境,于是想写一个宏包来封装,结合 \cls{xdyy-notes} 文类 \footnote{
  https://github.com/xkwxdyy/xdyy-math \\
  https://gitee.com/xkwxdyy/xdyy-math
}
一起使用


\section{命令介绍}


\subsection{推导命令 \tn{deduce} }

\begin{function}[added = 2022-1-12]{\deduce}
  \begin{syntax}
    \tn{deduce} \oarg{\kvopt{key}{val}} \marg{left content} \marg{right content}
  \end{syntax}
  思路推导命令,内容部分可以使用 \env{enumerate}、\env{itemize} 等环境
  \begin{hexample}
    \deduce{
      \begin{itemize}
        \item $3 > 2$
        \item $2 > 1$
        \item $1 > 0$
      \end{itemize}
    }{
      \begin{enumerate}
        \item $3 > 1$
        \item $2 > 0$
      \end{enumerate}
    }
  \end{hexample}
\end{function}

\tn{deduce} 命令的键值有:

\begin{function}[added = 2022-01-12]{arrow}
  \begin{syntax}
    |arrow| = \meta{箭头样式} \init{$\Longrightarrow$}
  \end{syntax}
  命令中间箭头的样式
\end{function}

\begin{function}[added = 2022-01-12]{leftbox-align, rightbox-align}
  \begin{syntax}
    |leftbox-align| = \meta{left, center, right} \init{right}
    |rightbox-align| = \meta{left, center, right} \init{left}
  \end{syntax}
  分别对应左边和右边内容的对齐方式,如果使用了 \env{enumerate} 或 \env{itemize} 等环境则不起作用,只对纯文本并且使用 \textbackslash \textbackslash 换行产生作用(如果只有一行的话用对齐也没啥太明显的效果hhh)。
\end{function}

\begin{function}[added = 2022-01-12]{show-leftbrace, show-rightbrace}
  \begin{syntax}
    |show-leftbrace| = \meta{true, false} \init{true}
    |show-rightbrace| = \meta{true, false} \init{true}
  \end{syntax}
  是否显示两边的花括号。
\end{function}

\begin{function}[added = 2022-01-12]{leftbrace-to-arrow, arrow-to-rightbrace}
  \begin{syntax}
    |leftbrace-to-arrow| = \meta{dimension} \init{7pt}
    |arrow-to-rightbrace| = \meta{dimension} \init{2pt}
  \end{syntax}
  顾名思义,分别是左括号到箭头的水平距离与箭头到右括号的水平距离。
\end{function}

\begin{function}[added = 2022-01-12]{rightbrace-to-rightbox}
  \begin{syntax}
    |rightbrace-to-rightbox| = \meta{dimension} \init{3mm}
  \end{syntax}
  顾名思义,右边的括号到右边内容盒子的水平距离。(因为 \env{enumerate} 或 \env{itemize} 等环境的左侧会有一段小距离,所以设置此键值控制,因为这些环境的右边并没有额外距离,所以没有设置左边括号到左边内容盒子的(如果有需求的话后续再设置))。
\end{function}



\section{环境介绍}


\subsection{定理类环境}

常见的定理环境均定义:
常见的定理环境
  \begin{xchoices}[label-style = arabic]
    \item 定义
    \item 定理
    \item 例
    \item 性质
    \item 命题
    \item 推论
    \item 引理
    \item 公理
    \item 注
    \item 反例
    \item 提示
    \item 总结
    \item 分析
    \item method方法(list型环境)
    \item step步骤(list型环境)
    \item case多情形(list型环境)
  \end{xchoices}

\begin{vexample}
    \begin{definition}[Euclid定理]
      这是一段文字 $E = m c^2$
    \end{definition}
    
    \begin{theorem}[Euclid定理]
      这是一段文字 $E = m c^2$
    \end{theorem}
    
    \begin{example}[Euclid定理]
      这是一段文字 $E = m c^2$
    \end{example}
    
    \begin{property}[Euclid定理]
      这是一段文字 $E = m c^2$
    \end{property}
    
    \begin{proposition}[Euclid定理]
      这是一段文字 $E = m c^2$
    \end{proposition}
\end{vexample}

\begin{vexample}

    \begin{corollary}[Euclid定理]
      这是一段文字 $E = m c^2$
    \end{corollary}
    
    \begin{lemma}[Euclid定理]
      这是一段文字 $E = m c^2$
    \end{lemma}
    
    \begin{axiom}[Euclid定理]
      这是一段文字 $E = m c^2$
    \end{axiom}
    
    \begin{antiexample}[Euclid定理]
      这是一段文字 $E = m c^2$
    \end{antiexample}
    
    \begin{remark}
      这是一段文字 $E = m c^2$
    \end{remark}
    
    \begin{hint}
      这是一段文字 $E = m c^2$
    \end{hint}
    
    \begin{summary}
      这是一段文字 $E = m c^2$
    \end{summary}
\end{vexample}


\begin{vexample}
    \begin{analysis}
      这是一段文字 $E = m c^2$
    \end{analysis}
    
    \begin{proof}[Euclid定理的证明]
      \begin{method}
        \item 方法一
        \item 方法二
      \end{method}
    \end{proof}
    
    \begin{proof}
      文字
      \begin{method}
        \item 方法一
        \item 方法二
      \end{method}
    \end{proof}
\end{vexample}


\begin{vexample}
    \begin{proof}
      \begin{case}
        \item 情形1
        \item 情形2
      \end{case}
    \end{proof}
    
    \begin{proof}
      文字
      \begin{case}
        \item 情形1
        \item 情形2
      \end{case}
    \end{proof}
\end{vexample}

\begin{vexample}
    \begin{proof}
      \begin{step}
        \item 步骤1
        \item 步骤2
      \end{step}
    \end{proof}
    
    \begin{proof}
      文字
      \begin{step}
        \item 步骤1
        \item 步骤2
      \end{step}
    \end{proof}
\end{vexample}
\end{document}
