\documentclass{xdyy-usermanual}

\xdyymanualsetup{
  info = {
    author            = {夏康玮},
    title             = { \pkg{xdyy-math} 用户手册},
    email             = {kangweixia_xdyy@163.com},
    date              = {2022-02-16},
    version           = {0.0.1},
    github-repository = {https://github.com/xkwxdyy/xdyy-math},
    gitee-repository  = {https://gitee.com/xkwxdyy/xdyy-math},
  }
}

\usepackage{xdyy-math}


\begin{document}
\maketitle
\tableofcontents


\section{宏包简介}

自己在记数学笔记的过程中难免会遇到写一些新命令环境,于是想写一个宏包来封装,原本写成了\cmd{cls}文类的形式,但是不够自由,有不小的局限性,于是改成了 \cmd{sty} 宏包的形式。


\section{命令介绍}

\subsection{推导命令\tn{deduce}}
\begin{function}[added = 2022-1-12]{\deduce}
  \begin{syntax}
    \tn{deduce} \oarg{\kvopt{key}{val}} \marg{left content} \marg{right content}
  \end{syntax}
  思路推导命令,内容部分可以使用 \env{enumerate}、\env{itemize} 等环境
  \begin{hexample}
    \deduce{
      \begin{itemize}
        \item $3 > 2$
        \item $2 > 1$
        \item $1 > 0$
      \end{itemize}
    }{
      \begin{enumerate}
        \item $3 > 1$
        \item $2 > 0$
      \end{enumerate}
    }
  \end{hexample}
\end{function}

\tn{deduce}命令的键值有:

\begin{function}[added = 2022-01-12]{arrow}
  \begin{syntax}
    |arrow| = \meta{箭头样式} \init{$\Longrightarrow$}
  \end{syntax}
  命令中间箭头的样式
\end{function}

\begin{function}[added = 2022-01-12]{leftbox-align, rightbox-align}
  \begin{syntax}
    |leftbox-align| = \meta{left, center, right} \init{right}
    |rightbox-align| = \meta{left, center, right} \init{left}
  \end{syntax}
  分别对应左边和右边内容的对齐方式,如果使用了 \env{enumerate} 或 \env{itemize} 等环境则不起作用,只对纯文本并且使用\textbackslash \textbackslash 换行产生作用(如果只有一行的话用对齐也没啥太明显的效果hhh)。
\end{function}

\begin{function}[added = 2022-01-12]{show-leftbrace, show-rightbrace}
  \begin{syntax}
    |show-leftbrace| = \meta{true, false} \init{true}
    |show-rightbrace| = \meta{true, false} \init{true}
  \end{syntax}
  是否显示两边的花括号。
\end{function}

\begin{function}[added = 2022-01-12]{leftbrace-to-arrow, arrow-to-rightbrace}
  \begin{syntax}
    |leftbrace-to-arrow| = \meta{dimension} \init{7pt}
    |arrow-to-rightbrace| = \meta{dimension} \init{2pt}
  \end{syntax}
  顾名思义,分别是左括号到箭头的水平距离与箭头到右括号的水平距离。
\end{function}

\begin{function}[added = 2022-01-12]{rightbrace-to-rightbox}
  \begin{syntax}
    |rightbrace-to-rightbox| = \meta{dimension} \init{3mm}
  \end{syntax}
  顾名思义,右边的括号到右边内容盒子的水平距离。(因为 \env{enumerate} 或 \env{itemize} 等环境的左侧会有一段小距离,所以设置此键值控制,因为这些环境的右边并没有额外距离,所以没有设置左边括号到左边内容盒子的(如果有需求的话后续再设置))。
\end{function}

\end{document}
